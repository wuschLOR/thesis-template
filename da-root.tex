%%%%%%%%%%%%%%%%%%%%
%%%%% Hinweise %%%%%
%%%%%%%%%%%%%%%%%%%%
%
% Kubuntu 12.04
% Schritt 1:
%   Hinzufügen der von ppa:texlive-backports/ppa
%   Anleitung findet sich hier: http://wiki.ubuntuusers.de/Tex_Live
% Schritt 2:
%   falls biber nicht installiert wurde das ganze noch per Hand durch das Debinapaket hinzufügen
%   Download: https://packages.debian.org/wheezy/all/biber/download
%   und dann einfach ausführen
% Schritt 3: (wenn kile verwendet wird)
%   Konfiguration: Settings -> Configure Kile -> Tools -> Build
%   Hinzufügen von Biber: New
%   Konfiguration von Quickbuild:         PDFLaTeX
%                                         biber
%                                         PDFLaTeX
% Schritt 4:
%   Beten bzw. feiern das es klappt

%%%%%%%%%%%%%%%%%%%%%%%%%%%%%%
%%%%% Arbeiten mit LATEX %%%%%
%%%%%%%%%%%%%%%%%%%%%%%%%%%%%%
%
% Hier einige Tips für das Arbeiten mit LaTeX:
%   Die Dokumentation ist dein Freund:
%     LaTeX benutzt Pakete und Skripte die das arbeiten erleichtern.
%     Diese werden meist mit \usepackage[options]{paket name} geladen.
%     Die Dokumentation findet man auf: http://www.ctan.org/ .
%     In der Dokumentation stehen die Optionen die man in die eckigen Klammern schreiben kann - was je nach Paket und Option krasse unterscheide im Verhalten und somit auch im Aussehen des Dokuments hat.
%     WICHTIG: Erst die Doku anschauen dann Fragen stellen
%   Fehler
%     Hat man ein bisschen zu viel gebastelt sollte man sich unbedingt das log file anschauen - hier ist meist aufgeführt welche Probleme es gibt
%   Fragen
%     Wenn es nicht mehr weiter geht gibt es zwei Adressen an die man sich auf jeden fall wenden kann (Abgesehen von Google)
%     https://tex.stackexchange.com
%     #latex freenode
  

%%%%%%%%%%%%%%%%%%%%
%%%%% Settings %%%%%
%%%%%%%%%%%%%%%%%%%%

% ist hier weil sonnst kile rummotzt -  rest ist in den sttings
%%%%% KOMA report %%%%
\documentclass[
  paper=a4,
  fontsize=12pt
]{scrreprt}

\author{Michael Groh}
\title{Formwahrnehmung:Blinker und ihre impliziten Richtungshinweise bei Automobilen}

%Alle Einstellung sind hier drin
% %%%%% KOMA report %%%% - ausglagert in root
% \documentclass[
%   paper=a4,
%   fontsize=12pt
% ]{scrreprt}


%%%%% KOMA hack %%%%%
% da das KOMA Skript einen Fehler ausgibt da irgendwas mit floar@addtolist böse ist - macht dass der fahler ignoriert wird ?
% https://tex.stackexchange.com/questions/51867/koma-warning-about-toc
\usepackage{scrhack} 

%%%%% KOMA Hierarchie scrreprt %%%%%
% \chapter[short version ]{heading }
% \section[short version ]{heading }
% \subsection[short version ]{heading }
% \subsubsection[short version ]{heading }
% \paragraph[short version ]{heading }
% \subparagraph[short version ]{heading }

\usepackage[ngerman]{babel} 
\usepackage[utf8]{inputenc}
\usepackage[T1]{fontenc}        %https://tex.stackexchange.com/questions/4268/inputenc-error-unicode-char-u8-error-while-trying-to-write-a-degree-symbol
\usepackage{textcomp}           %https://tex.stackexchange.com/questions/4268/inputenc-error-unicode-char-u8-error-while-trying-to-write-a-degree-symbol
\usepackage{gensymb}

%%%% TABELLEN Überschriften %%%%%
\usepackage[
  size     = footnotesize,
  nooneline,
  format   = plain,
%   labelsep = period
]{caption}  % <=== NEU
% captions für table und figure definieren
% https://tex.stackexchange.com/questions/186228/2-different-captions-for-tables-and-figures
\captionsetup[figure]{labelsep = quad}
\captionsetup[table]{labelsep = newline}

%%%%% LINKS %%%%%
% macht links im ganzen Dokument nötig
% http://www.ctan.org/pkg/hyperref
% ABHÄNGIG
%       biblatex
% https://en.wikibooks.org/wiki/LaTeX/Hyperlinks#Customizationl
\usepackage[
  colorlinks=true,
%   hidelinks, %printig
  linkcolor=red,
  urlcolor=blue
]{hyperref}

%%%%% ZITATE Anführungszeichen %%%%%
% Anführungszeichen sind jetzt einheilich
% https://de.wikibooks.org/wiki/LaTeX-W%C3%B6rterbuch:_Anf%C3%BChrungszeichen
% http://www.ctan.org/pkg/csquotes
\usepackage[
  babel,
  german=quotes
]{csquotes}

%%%%% ZITATE Spezifikation und APA style %%%%%
\usepackage[
  backend=biber,        %use biber
  style=apa,            %use biblatex-apa 
  sortcites=true,       % sortiere bei mehrfachen Zitationen 
  sorting=nyt,          %sortiere das Literaturverzeichnis nyt Sort by name, year, title.
  hyperref=true,        % benutzt paket hyperref und macht dass links links sind - erkennt auch doi ^^
  backref=true,         % macht dass refferenzen verlinkt werden also man lustig im dokument rumspringen kann
 % date=iso8601          %
  alldates=iso8601      % nachteil alle datum dineger werden im litertaturverzeichnisageben also was vorher 2014 wer ist jetzt 2014-02 wenn das angegeben war 
]{biblatex}

% & Der APA stil schereibt dafault vor, dass Titel komplett klein geschrieben werden - da das aber quatsch isf für deutsche titel müssen alle titel die groß und kleinschreibung erhalten sollen in {{}} eingefasst werden - es ist auch möglich z.B. bei Eigennamen nur den buchstaben in {} zuetzten
%
% Beispiel:
%   title = {Alle meine Entchen} -> Alle meine entchen
%   title = {{Alle meine Entchen}} -> Alle meine Entchen
%   title = {Alle meine {E}ntchen} -> Alle meine Entchen

%%%%% ???? %%%%%
%http://www.golatex.de/biblatex-apa-undefinded-mkbibdateapalongextra-t9943.html
\DeclareLanguageMapping{german}{german-apa} 


% et al
\DefineBibliographyStrings{ngerman}{
   andothers = {{et\,al\adddot}},            
} 


% Silbentrennung wenn möglich vermeiden
\sloppy

% Hurenkinder und Schusterjungen verhindern
\clubpenalty10000
\widowpenalty10000
\displaywidowpenalty=10000
\usepackage{etoolbox}
\makeatletter
\patchcmd{\@makechapterhead}{50\p@}{25pt}{}{}
\patchcmd{\@makeschapterhead}{50\p@}{25pt}{}{}
\makeatother


%Tabellen
\usepackage{longtable}
\usepackage{booktabs,tabularx} % tabularx ersetze l c r bei tabellen durch X - macht dann fancy automatischen Zeilenumbruch
\usepackage{ctable} % needed for \cmidrule{}
\usepackage{multirow} % needed for \multirow{}
\newcommand{\head}[1]{\textnormal{\textbf{#1}}} %zusatzbefahl macht nur dass die entsprechende zeile fett ist (nach apa)



\usepackage{listings}


\usepackage{graphicx}

%tabellenmagie
% https://tex.stackexchange.com/questions/192186/given-a-3x4-table-of-images-how-do-i-make-all-the-images-the-same-size-and-remo#192200
\newcommand{\x}{0.05}
\newcommand\ig[1][]{\includegraphics[width= \x\linewidth, height = \x\linewidth,keepaspectratio=true,#1]}


\usepackage{framed}

%kapiteweise nummerierung
% \usepackage{amsmath}
\usepackage{mathtools}
\numberwithin{figure}{chapter}
\numberwithin{table}{chapter}



%math symbols
\usepackage{amsfonts}
\usepackage{amssymb}



%%%%%%%%%%%%%%%%%%%%%%%%%%%%%%%%%%%%%%%%greek itallic
% https://tex.stackexchange.com/questions/4610/package-for-changing-font-with-each-letter
% \newfontfamily\fontg{TeX Gyre Termes}
%% ==================> no working with pdf latex

% https://tex.stackexchange.com/questions/30049/how-to-select-math-font-in-document
% \setmathfont[version=termes]{TeX Gyre Termes Math}
%% ==================> no working with pdf latex

% \usepackage[uprightgreeks]{kpfonts} %coole schriftart
%% ==================> der gesammte text wird anders

% \usepackage[LGRgreek]{mathastext}
%% ==================> everything is upright

% https://tex.stackexchange.com/questions/12664/using-mathdesigns-math-font-with-another-text-font
% \usepackage[
%   greekfamily=bodoni,
%   greekuppercase=upright,
%   greeklowercase=upright
% ]{mathdesign}
%% ==================> no change at all

\usepackage{upgreek}


% \DeclareMathSymbol{\chi}{7}{operators}{1F}
% \DeclareMathSymbol{\chi}{\mathord}{operators}{\chi}
% \DeclareMathSymbol{\Delta}{\mathalpha}{operators}{1}
% \DeclareMathSymbol{\Theta}{\mathalpha}{operators}{2}
% \DeclareMathSymbol{\Lambda}{\mathalpha}{operators}{3}
% \DeclareMathSymbol{\Xi}{\mathalpha}{operators}{4}
% \DeclareMathSymbol{\Pi}{\mathalpha}{operators}{5}
% \DeclareMathSymbol{\Sigma}{\mathalpha}{operators}{6}
% \DeclareMathSymbol{\Upsilon}{\mathalpha}{operators}{7}
% \DeclareMathSymbol{\Phi}{\mathalpha}{operators}{8}
% \DeclareMathSymbol{\Psi}{\mathalpha}{operators}{9}
% \DeclareMathSymbol{\Omega}{\mathalpha}{operators}{10}

% \usepackage{fixmath} %macht dass \mathrm{} bei griechen funtzt


\usepackage{longtable} %tabs mit seitenumbruch % geht nicht

\usepackage{pdflscape} % seite kippen für tabelle

%damit ° in math mode erkannt wird
% https://tex.stackexchange.com/questions/84469/unicode-degree-symbol-%C2%B0-in-math-formulas-textdegree-invalid-in-math-mode
%\usepackage{newunicodechar}
%\newunicodechar{°}{\degree}

\usepackage{array}
\usepackage{url}

%Pdfs einbinden
\usepackage{pdfpages}



\usepackage{geometry}
% \graphicspath{ {./abb/} } % Pfad für die Bilder

\geometry{a4paper,left=3cm, right=3cm, top=3.7cm, bottom=3.9cm, bindingoffset=0.2cm} % Vorgabe APA

\linespread{1.25} % eineinhalb-facher Zeilenabstand

%Kopfzeile
\usepackage{fancyhdr}
\setlength{\headheight}{15pt} %http://nw360.blogspot.de/2006/11/latex-headheight-is-too-small.html



%hypothesen
% http://tex.stackexchange.com/questions/100880/hypotheses-and-subhypotheses-with-ntheorem
\usepackage[hyperref]{ntheorem}
%\makeatletter
  \theoremstyle{break}
  \newtheorem{mainhyp}{Fragestellung}
  \providecommand*{\mainhypautorefname}{Fragestellung}


  \newtheorem{hyp}{Hypothese}
  \providecommand*{\hypautorefname}{Hypothese} %sagt hyperref was es anzeigen soll
  % use \autoref indead of thref (cause you don't want to use different refs)
  % http://tex.stackexchange.com/questions/24568/autoref-names-messed-up-when-defining-certain-ntheorem-theoremstyles/24923#24923
  
%\makeatother

%acronym usage
\usepackage[
  printonlyused,
  withpage 
]{acronym}

\addtokomafont{disposition}{\rmfamily} %überschriften times



\usepackage{lipsum} % um dummy text hinzuklatschen 



\usepackage{afterpage}
% (16:51:58) |Zz|: then try this \usepackage{afterpage}
% (16:52:15) |Zz|: and then put \afterpage{\clearpage} just after that "big figure"
% (16:52:27) |Zz|: it should appear in the next available page, instead at the end of the chapter
% (16:52:34) |Zz|: and that will not disrupt the following figures
% (16:52:41) |Zz|: repeat that trick for each "big figure"

% für unter /labels
\usepackage[list=true,listformat=simple]{subcaption}
\graphicspath{ {./abb/} {./abbShapes1/} {./abbShapes2/} {./abbShapes3/} {./append/} {./shapes/} } % Pfad für die Bilder

%einbinden der Literaturdatenbank (biblatex)
  %Details in settings.txt
\addbibresource{./bib/biblatex.bib}





%%%%%%%%%%%%%%%%%%%%
%%%%% DOKUMENT %%%%%
%%%%%%%%%%%%%%%%%%%%
% Hier beginnt das Dokument 
\begin{document}

%%%%%%%%%%%%%%%%%%%%%%%%%%%%%%%%%%%%%%%%%
%%%%% Sachen die vorher rein müssen %%%%%
%%%%%%%%%%%%%%%%%%%%%%%%%%%%%%%%%%%%%%%%%

% Titelseite



\thispagestyle{empty}
\vspace{4cm}
\begin{center}
\textbf{\Large{Ägyptischen Pyramiden}} \\
\vspace{1cm}
\textbf{\large{als Landestellen für extraterrestriale Raumschiffe}}
\vspace{3cm}



\Large{\textbf{Dissertation}} 

\vspace{0.5cm}
im Studiengang Archeologie
in der\\
Fakultät Geistes- und Kulturwissenschaften an der\\
University of Chicago\\
(2015)
\end{center}

\vspace {5cm}

\begin{tabular}{l l}
  Verfasser: & Daniel Jackson\\ 
  Korrektor: & Dr. David Jordan \\ 
\end{tabular} 

\pagebreak

% Seitenstil für Erste Seiten definieren
\pagestyle{plain}
\setcounter{page}{1}
\pagenumbering{Roman}

% Danksagungen
% \input{./tex/dank.tex}
% \pagebreak

% Hier die Abstracts
\input{./tex/abstract.tex}
\pagebreak

%Inhaltsverzeichnis
\tableofcontents
\addcontentsline{toc}{chapter}{Inhaltsverzeichnis}
\pagebreak

% plain-Seitenstil umdefinieren (Kapitel-Anfangsseiten)
\fancypagestyle{plain}{%
\fancyhf{}\renewcommand{\headrulewidth}{0pt}}

% Seitenstil für Arbeit definieren
\pagestyle{fancy}
\renewcommand{\chaptermark}[1]%
{\markboth{#1}{}}

\fancyhead{}
\fancyfoot{}

\lhead{\leftmark}
\rhead{\thepage}


\setcounter{page}{1}
\pagenumbering{arabic}

%%%%%%%%%%%%%%%%%%%
%%%%% Kapitel %%%%%
%%%%%%%%%%%%%%%%%%%

%blueprint chapter
\chapter{blueprint chapter}

%\pagebreak

\section{section}
teyt teyt mininiu
\subsection{subsection}

\lipsum[1]

\subsubsection{subsubsection}

\paragraph{paragraph}

\subparagraph{subparagraph}

\section{citing}
% Zitieren
\begin{itemize}
  \item autocite  \autocite {zongker_chicken_2006}
  \item textcite  \textcite {zongker_chicken_2006}
  \item parencite \parencite{zongker_chicken_2006}
\end{itemize}

immer parencite verwenden weil das halt teil vom apa paket is :)

\section{figures}

%Abbildungen (APA like)
\begin{figure}[tb]
  \centering
  \includegraphics[width=\linewidth,
                   height=0.33\textheight,
                   keepaspectratio=true]{cat.jpeg} 
  %\captionsetup{labelsep=period}
  \caption[Katze am Weihnachtsbaum (das steht im abbverzeichnis)]{
    \textit{Katze am Weihnachtsbaum (Abbildungsbeschriftung)}
    Alle weiteren Sätze
    }
  \label{fig:cat}
\end{figure}

\autoref{fig:cat} refferiert auf die grafik



\section{tabels}

%Tabellen (APA like)
\begin{table}[tb]
  \caption[das steht im tabverzeichnis]{\textit{Tabbeschriftung}}
  \begin{flushleft}
    \footnotesize
    \centering
    \begin{tabular}{ccccccc}
      \toprule
          & \multicolumn{2}{c}{T0} & \multicolumn{2}{c}{T1} & \multicolumn{2}{c}{T2}\\
      \cmidrule(lr){2-7} 
          & M    & SD   & M    & SD   & M    & SD   \\
      \hline
      Alle & 2.98 & 1.66 & 3.73 & 1.29 & 0.75 & 1.49 \\
      \hline
      KG   & 2.31 & 1.45 & 3.49 & 1.27 & 1.18 & 1.55 \\
      EG   & 3.87 & 1.5  & 4.05 & 1.24 & 0.19 & 1.19 \\
      \bottomrule
    \end{tabular}
  \end{flushleft}
  \label{table:Zeug}
\end{table}

\section{hypothesen}
\begin{hyp}[mit krassem Inhalt]
\label{hyp:einleitung}
The better the answer, the higher the score.
\end{hyp}


blabla itermetzo

\autoref{hyp:einleitung} wurde super bestätigt lalalalalalala


% \section{acronyms}
% 
% 
% \ac{KDE} Gibt bei der ersten Verwendung die Langform mit der Abkürzung in Klammern aus, ab dann stets die Kurzform.
% 
% 
% \acs{KDE} Gibt die Abkürzung aus.
% 
% 
% \acf{KDE} Gibt die Langform und die Kurzform aus.
% 
% 
% \acl{KDE} Gibt nur die Langform ohne die Kurzform aus.



\section{crazy minipage or tabular or subcapition}
\begin{figure}
  \centering
  \begin{tabular}{p{.45\textwidth}p{.45\textwidth}}
      \includegraphics[width=\linewidth,
                       height=0.33\textheight,
                       keepaspectratio=true]{cat.jpeg} 
      %\captionsetup{labelsep=period}
      \caption[Katze am Weihnachtsbaum (das steht im abbverzeichnis)]{
        \textit{Katze am Weihnachtsbaum (Abbildungsbeschriftung)}
        Alle weiteren Sätze
        }
      \label{fig:cat2}
    &
      \includegraphics[width=\linewidth,
                       height=0.33\textheight,
                       keepaspectratio=true]{cat.jpeg} 
      %\captionsetup{labelsep=period}
      \caption[Katze am Weihnachtsbaum (das steht im abbverzeichnis)]{
        \textit{Katze am Weihnachtsbaum (Abbildungsbeschriftung)}
        Alle weiteren Sätze
        }
      \label{fig:cat3}
    \\
  \end{tabular}
\end{figure}


% \begin{figure}
%   \centering
%   \begin{tabular}{p{.45\textwidth}p{.45\textwidth}}
%     &
%     \\
%   \end{tabular}
% \end{figure}

oder 

\begin{figure}[htb]
  \begin{minipage}[t]{0.45\linewidth}
    \centering
      \includegraphics[width=\linewidth,
                       height=0.33\textheight,
                       keepaspectratio=true]{cat.jpeg} 
      %\captionsetup{labelsep=period}
      \caption[Katze am Weihnachtsbaum (das steht im abbverzeichnis)]{
        \textit{Katze am Weihnachtsbaum (Abbildungsbeschriftung)}
        Alle weiteren Sätze
        }
      \label{fig:cat4}
  \end{minipage}
  \hspace{0.05\linewidth}
  \begin{minipage}[t]{0.45\linewidth}
    \centering
      \includegraphics[width=\linewidth,
                       height=0.33\textheight,
                       keepaspectratio=true]{cat.jpeg} 
      %\captionsetup{labelsep=period}
      \caption[Katze am Weihnachtsbaum (das steht im abbverzeichnis)]{
        \textit{Katze am Weihnachtsbaum (Abbildungsbeschriftung)}
        Alle weiteren Sätze
        }
      \label{fig:cat5}
  \end{minipage}
%   \par 
%   \begin{minipage}[t]{0.45\linewidth}
% 
%   \end{minipage}
%   \hspace{0.05\linewidth}
%   \begin{minipage}[t]{0.45\linewidth}
% 
%   \end{minipage}
\end{figure}


oder



\begin{figure}
  \label{fig:subfig_main}
  \begin{subfigure}[c]{0.45\linewidth}
    \label{fig:subfig_1}
    \includegraphics[height=0.33\textheight,
                     keepaspectratio=true]{cat}
    \subcaption{Subfigure Bild Nr. 1}
  \end{subfigure}
  \begin{subfigure}[c]{0.45\linewidth}
    \label{fig:subfig_2}
    \includegraphics[height=0.33\textheight,
                     keepaspectratio=true]{cat}
    \subcaption{Subfigure Bild Nr. 2}
  \end{subfigure}
  \caption{Zwei Bilder mit Subfigure nebeneinander}
\end{figure}




wenn zeit ist das ganze auch mal mit subfloats machen 
\url{https://en.wikibooks.org/wiki/LaTeX/Floats,_Figures_and_Captions#Subfloats}{party party tako borito}

GROUP1 $ M=2.671, SD=1.065 $
GROUP2 $ M=4.346, SD=1.705 $
T-TEST $ t(14)=-3.072,p<.01,d=0.793 $


kijfdsaaaaaaaaaaaaaaaaaaaaaaaaaaaaaaa \autoref{fig:subfig_2} \ref{fig:subfig_2} \autoref{fig:subfig_main} \ref{fig:subfig_main}
\pagebreak

% Einleitung
\input{./tex/einleitung.tex}
\pagebreak

%Fragestellung
\input{./tex/fragestellung.tex}
\pagebreak

%Theorie Block
\input{./tex/theorie.tex}
\pagebreak

%Experiment Aufbau und so
\input{./tex/experiment.tex}
\pagebreak

%Ergebnisse
\input{./tex/ergebnisse.tex}
\pagebreak

%Diskussion und Ausblick
\input{./tex/discussion.tex}
\pagebreak


%%%%%%%%%%%%%%%%%%%%%%%%%
%%%%% Verzeichnisse %%%%%
%%%%%%%%%%%%%%%%%%%%%%%%%

% Hypothesen und Fragestellungen
\chapter*{Thesenverzeichnis}
\addcontentsline{toc}{chapter}{Thesenverzeichnis}

\section{Fragestellungen}
\listtheorems{mainhyp}
% \section{Subfragestellungen}
\section{Versuchshypothesen}
\listtheorems{hyp}



%Abbildungsverzeichnis
\listoffigures
\addcontentsline{toc}{chapter}{Abbildungsverzeichnis}
\pagebreak

%Tabellenverzeichnis
\listoftables
\addcontentsline{toc}{chapter}{Tabellenverzeichnis}
\pagebreak


% https://tex.stackexchange.com/questions/41818/section-added-by-addcontentsline-on-the-wrong-page
% With this modification we doesn't have an entry in the toc. To provide an entry in the toc we can use the option heading. biblatexoffers different values for this option. Here a small range:
% 
%     bibliography:
%         unnumbered -- no entry in toc -- chapter level (section for articles)
%     subbibliography:
%         unnumbered -- no entry in toc -- section level (subsection for articles)
%     bibintoc:
%         unnumbered -- entry in toc -- chapter level (section for articles)
%     subbibintoc:
%         unnumbered -- entry in toc -- section level (subsection for articles)
%     bibnumbered:
%         numbered -- entry in toc -- chapter level (section for articles)
%     subbibnumbered:
%         numbered -- entry in toc -- section level (subsection for articles)


%Softwareverzeichnis
\printbibliography[
  heading= bibintoc,
  title={Softwareverzeichnis},
  type=software
]
% \addcontentsline{toc}{chapter}{Softwareverzeichnis}

%Literaturverzeichnis
\printbibliography[
  heading= bibintoc,
  title={Literaturverzeichnis},
  nottype=software, %steht im Softwareverzeichnis
]
% \addcontentsline{toc}{chapter}{Literaturverzeichnis}
\pagebreak



%Anhang
\input{./tex/appendix.tex}
\pagebreak

%%%%%%%%%%%%%%%%%%%%%
%%%%% Erklärung %%%%%
%%%%%%%%%%%%%%%%%%%%%

%Abschlusserklärung
\input{./tex/erklaerung.tex}

\end{document}