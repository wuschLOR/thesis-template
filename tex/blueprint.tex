\chapter{blueprint chapter}

%\pagebreak

\section{section}
teyt teyt mininiu
\subsection{subsection}

\lipsum[1]

\subsubsection{subsubsection}

\paragraph{paragraph}

\subparagraph{subparagraph}

\section{citing}
% Zitieren
\begin{itemize}
  \item autocite  \autocite {zongker_chicken_2006}
  \item textcite  \textcite {zongker_chicken_2006}
  \item parencite \parencite{zongker_chicken_2006}
\end{itemize}

immer parencite verwenden weil das halt teil vom apa paket is :)

\section{figures}

%Abbildungen (APA like)
\begin{figure}[tb]
  \centering
  \includegraphics[width=\linewidth,
                   height=0.33\textheight,
                   keepaspectratio=true]{cat.jpeg} 
  %\captionsetup{labelsep=period}
  \caption[Katze am Weihnachtsbaum (das steht im abbverzeichnis)]{
    \textit{Katze am Weihnachtsbaum (Abbildungsbeschriftung)}
    Alle weiteren Sätze
    }
  \label{fig:cat}
\end{figure}

\autoref{fig:cat} refferiert auf die grafik



\section{tabels}

%Tabellen (APA like)
\begin{table}[tb]
  \caption[das steht im tabverzeichnis]{\textit{Tabbeschriftung}}
  \begin{flushleft}
    \footnotesize
    \centering
    \begin{tabular}{ccccccc}
      \toprule
          & \multicolumn{2}{c}{T0} & \multicolumn{2}{c}{T1} & \multicolumn{2}{c}{T2}\\
      \cmidrule(lr){2-7} 
          & M    & SD   & M    & SD   & M    & SD   \\
      \hline
      Alle & 2.98 & 1.66 & 3.73 & 1.29 & 0.75 & 1.49 \\
      \hline
      KG   & 2.31 & 1.45 & 3.49 & 1.27 & 1.18 & 1.55 \\
      EG   & 3.87 & 1.5  & 4.05 & 1.24 & 0.19 & 1.19 \\
      \bottomrule
    \end{tabular}
  \end{flushleft}
  \label{table:Zeug}
\end{table}

\section{hypothesen}
\begin{hyp}[mit krassem Inhalt]
\label{hyp:einleitung}
The better the answer, the higher the score.
\end{hyp}


blabla itermetzo

\autoref{hyp:einleitung} wurde super bestätigt lalalalalalala


% \section{acronyms}
% 
% 
% \ac{KDE} Gibt bei der ersten Verwendung die Langform mit der Abkürzung in Klammern aus, ab dann stets die Kurzform.
% 
% 
% \acs{KDE} Gibt die Abkürzung aus.
% 
% 
% \acf{KDE} Gibt die Langform und die Kurzform aus.
% 
% 
% \acl{KDE} Gibt nur die Langform ohne die Kurzform aus.



\section{crazy minipage or tabular or subcapition}
\begin{figure}
  \centering
  \begin{tabular}{p{.45\textwidth}p{.45\textwidth}}
      \includegraphics[width=\linewidth,
                       height=0.33\textheight,
                       keepaspectratio=true]{cat.jpeg} 
      %\captionsetup{labelsep=period}
      \caption[Katze am Weihnachtsbaum (das steht im abbverzeichnis)]{
        \textit{Katze am Weihnachtsbaum (Abbildungsbeschriftung)}
        Alle weiteren Sätze
        }
      \label{fig:cat2}
    &
      \includegraphics[width=\linewidth,
                       height=0.33\textheight,
                       keepaspectratio=true]{cat.jpeg} 
      %\captionsetup{labelsep=period}
      \caption[Katze am Weihnachtsbaum (das steht im abbverzeichnis)]{
        \textit{Katze am Weihnachtsbaum (Abbildungsbeschriftung)}
        Alle weiteren Sätze
        }
      \label{fig:cat3}
    \\
  \end{tabular}
\end{figure}


% \begin{figure}
%   \centering
%   \begin{tabular}{p{.45\textwidth}p{.45\textwidth}}
%     &
%     \\
%   \end{tabular}
% \end{figure}

oder 

\begin{figure}[htb]
  \begin{minipage}[t]{0.45\linewidth}
    \centering
      \includegraphics[width=\linewidth,
                       height=0.33\textheight,
                       keepaspectratio=true]{cat.jpeg} 
      %\captionsetup{labelsep=period}
      \caption[Katze am Weihnachtsbaum (das steht im abbverzeichnis)]{
        \textit{Katze am Weihnachtsbaum (Abbildungsbeschriftung)}
        Alle weiteren Sätze
        }
      \label{fig:cat4}
  \end{minipage}
  \hspace{0.05\linewidth}
  \begin{minipage}[t]{0.45\linewidth}
    \centering
      \includegraphics[width=\linewidth,
                       height=0.33\textheight,
                       keepaspectratio=true]{cat.jpeg} 
      %\captionsetup{labelsep=period}
      \caption[Katze am Weihnachtsbaum (das steht im abbverzeichnis)]{
        \textit{Katze am Weihnachtsbaum (Abbildungsbeschriftung)}
        Alle weiteren Sätze
        }
      \label{fig:cat5}
  \end{minipage}
%   \par 
%   \begin{minipage}[t]{0.45\linewidth}
% 
%   \end{minipage}
%   \hspace{0.05\linewidth}
%   \begin{minipage}[t]{0.45\linewidth}
% 
%   \end{minipage}
\end{figure}


oder



\begin{figure}
  \label{fig:subfig_main}
  \begin{subfigure}[c]{0.45\linewidth}
    \label{fig:subfig_1}
    \includegraphics[height=0.33\textheight,
                     keepaspectratio=true]{cat}
    \subcaption{Subfigure Bild Nr. 1}
  \end{subfigure}
  \begin{subfigure}[c]{0.45\linewidth}
    \label{fig:subfig_2}
    \includegraphics[height=0.33\textheight,
                     keepaspectratio=true]{cat}
    \subcaption{Subfigure Bild Nr. 2}
  \end{subfigure}
  \caption{Zwei Bilder mit Subfigure nebeneinander}
\end{figure}




wenn zeit ist das ganze auch mal mit subfloats machen 
\url{https://en.wikibooks.org/wiki/LaTeX/Floats,_Figures_and_Captions#Subfloats}{party party tako borito}

GROUP1 $ M=2.671, SD=1.065 $
GROUP2 $ M=4.346, SD=1.705 $
T-TEST $ t(14)=-3.072,p<.01,d=0.793 $


kijfdsaaaaaaaaaaaaaaaaaaaaaaaaaaaaaaa \autoref{fig:subfig_2} \ref{fig:subfig_2} \autoref{fig:subfig_main} \ref{fig:subfig_main}