% %%%%% KOMA report %%%% - ausglagert in root
% \documentclass[
%   paper=a4,
%   fontsize=12pt
% ]{scrreprt}


%%%%% KOMA hack %%%%%
% da das KOMA Skript einen Fehler ausgibt da irgendwas mit floar@addtolist böse ist - macht dass der fahler ignoriert wird ?
% https://tex.stackexchange.com/questions/51867/koma-warning-about-toc
\usepackage{scrhack} 

%%%%% KOMA Hierarchie scrreprt %%%%%
% \chapter[short version ]{heading }
% \section[short version ]{heading }
% \subsection[short version ]{heading }
% \subsubsection[short version ]{heading }
% \paragraph[short version ]{heading }
% \subparagraph[short version ]{heading }

\usepackage[ngerman]{babel} 
\usepackage[utf8]{inputenc}
\usepackage[T1]{fontenc}        %https://tex.stackexchange.com/questions/4268/inputenc-error-unicode-char-u8-error-while-trying-to-write-a-degree-symbol
\usepackage{textcomp}           %https://tex.stackexchange.com/questions/4268/inputenc-error-unicode-char-u8-error-while-trying-to-write-a-degree-symbol
\usepackage{gensymb}

%%%% TABELLEN Überschriften %%%%%
\usepackage[
  size     = footnotesize,
  nooneline,
  format   = plain,
%   labelsep = period
]{caption}  % <=== NEU
% captions für table und figure definieren
% https://tex.stackexchange.com/questions/186228/2-different-captions-for-tables-and-figures
\captionsetup[figure]{labelsep = quad}
\captionsetup[table]{labelsep = newline}

%%%%% LINKS %%%%%
% macht links im ganzen Dokument nötig
% http://www.ctan.org/pkg/hyperref
% ABHÄNGIG
%       biblatex
% https://en.wikibooks.org/wiki/LaTeX/Hyperlinks#Customizationl
\usepackage[
  colorlinks=true,
%   hidelinks, %printig
  linkcolor=red,
  urlcolor=blue
]{hyperref}

%%%%% ZITATE Anführungszeichen %%%%%
% Anführungszeichen sind jetzt einheilich
% https://de.wikibooks.org/wiki/LaTeX-W%C3%B6rterbuch:_Anf%C3%BChrungszeichen
% http://www.ctan.org/pkg/csquotes
\usepackage[
  babel,
  german=quotes
]{csquotes}

%%%%% ZITATE Spezifikation und APA style %%%%%
\usepackage[
  backend=biber,        %use biber
  style=apa,            %use biblatex-apa 
  sortcites=true,       % sortiere bei mehrfachen Zitationen 
  sorting=nyt,          %sortiere das Literaturverzeichnis nyt Sort by name, year, title.
  hyperref=true,        % benutzt paket hyperref und macht dass links links sind - erkennt auch doi ^^
  backref=true,         % macht dass refferenzen verlinkt werden also man lustig im dokument rumspringen kann
 % date=iso8601          %
  alldates=iso8601      % nachteil alle datum dineger werden im litertaturverzeichnisageben also was vorher 2014 wer ist jetzt 2014-02 wenn das angegeben war 
]{biblatex}

% & Der APA stil schereibt dafault vor, dass Titel komplett klein geschrieben werden - da das aber quatsch isf für deutsche titel müssen alle titel die groß und kleinschreibung erhalten sollen in {{}} eingefasst werden - es ist auch möglich z.B. bei Eigennamen nur den buchstaben in {} zuetzten
%
% Beispiel:
%   title = {Alle meine Entchen} -> Alle meine entchen
%   title = {{Alle meine Entchen}} -> Alle meine Entchen
%   title = {Alle meine {E}ntchen} -> Alle meine Entchen

%%%%% ???? %%%%%
%http://www.golatex.de/biblatex-apa-undefinded-mkbibdateapalongextra-t9943.html
\DeclareLanguageMapping{german}{german-apa} 


% et al
\DefineBibliographyStrings{ngerman}{
   andothers = {{et\,al\adddot}},            
} 


% Silbentrennung wenn möglich vermeiden
\sloppy

% Hurenkinder und Schusterjungen verhindern
\clubpenalty10000
\widowpenalty10000
\displaywidowpenalty=10000
\usepackage{etoolbox}
\makeatletter
\patchcmd{\@makechapterhead}{50\p@}{25pt}{}{}
\patchcmd{\@makeschapterhead}{50\p@}{25pt}{}{}
\makeatother


%Tabellen
\usepackage{longtable}
\usepackage{booktabs,tabularx} % tabularx ersetze l c r bei tabellen durch X - macht dann fancy automatischen Zeilenumbruch
\usepackage{ctable} % needed for \cmidrule{}
\usepackage{multirow} % needed for \multirow{}
\newcommand{\head}[1]{\textnormal{\textbf{#1}}} %zusatzbefahl macht nur dass die entsprechende zeile fett ist (nach apa)



\usepackage{listings}


\usepackage{graphicx}

%tabellenmagie
% https://tex.stackexchange.com/questions/192186/given-a-3x4-table-of-images-how-do-i-make-all-the-images-the-same-size-and-remo#192200
\newcommand{\x}{0.05}
\newcommand\ig[1][]{\includegraphics[width= \x\linewidth, height = \x\linewidth,keepaspectratio=true,#1]}


\usepackage{framed}

%kapiteweise nummerierung
% \usepackage{amsmath}
\usepackage{mathtools}
\numberwithin{figure}{chapter}
\numberwithin{table}{chapter}



%math symbols
\usepackage{amsfonts}
\usepackage{amssymb}



%%%%%%%%%%%%%%%%%%%%%%%%%%%%%%%%%%%%%%%%greek itallic
% https://tex.stackexchange.com/questions/4610/package-for-changing-font-with-each-letter
% \newfontfamily\fontg{TeX Gyre Termes}
%% ==================> no working with pdf latex

% https://tex.stackexchange.com/questions/30049/how-to-select-math-font-in-document
% \setmathfont[version=termes]{TeX Gyre Termes Math}
%% ==================> no working with pdf latex

% \usepackage[uprightgreeks]{kpfonts} %coole schriftart
%% ==================> der gesammte text wird anders

% \usepackage[LGRgreek]{mathastext}
%% ==================> everything is upright

% https://tex.stackexchange.com/questions/12664/using-mathdesigns-math-font-with-another-text-font
% \usepackage[
%   greekfamily=bodoni,
%   greekuppercase=upright,
%   greeklowercase=upright
% ]{mathdesign}
%% ==================> no change at all

\usepackage{upgreek}


% \DeclareMathSymbol{\chi}{7}{operators}{1F}
% \DeclareMathSymbol{\chi}{\mathord}{operators}{\chi}
% \DeclareMathSymbol{\Delta}{\mathalpha}{operators}{1}
% \DeclareMathSymbol{\Theta}{\mathalpha}{operators}{2}
% \DeclareMathSymbol{\Lambda}{\mathalpha}{operators}{3}
% \DeclareMathSymbol{\Xi}{\mathalpha}{operators}{4}
% \DeclareMathSymbol{\Pi}{\mathalpha}{operators}{5}
% \DeclareMathSymbol{\Sigma}{\mathalpha}{operators}{6}
% \DeclareMathSymbol{\Upsilon}{\mathalpha}{operators}{7}
% \DeclareMathSymbol{\Phi}{\mathalpha}{operators}{8}
% \DeclareMathSymbol{\Psi}{\mathalpha}{operators}{9}
% \DeclareMathSymbol{\Omega}{\mathalpha}{operators}{10}

% \usepackage{fixmath} %macht dass \mathrm{} bei griechen funtzt


\usepackage{longtable} %tabs mit seitenumbruch % geht nicht

\usepackage{pdflscape} % seite kippen für tabelle

%damit ° in math mode erkannt wird
% https://tex.stackexchange.com/questions/84469/unicode-degree-symbol-%C2%B0-in-math-formulas-textdegree-invalid-in-math-mode
%\usepackage{newunicodechar}
%\newunicodechar{°}{\degree}

\usepackage{array}
\usepackage{url}

%Pdfs einbinden
\usepackage{pdfpages}



\usepackage{geometry}
% \graphicspath{ {./abb/} } % Pfad für die Bilder

\geometry{a4paper,left=3cm, right=3cm, top=3.7cm, bottom=3.9cm, bindingoffset=0.2cm} % Vorgabe APA

\linespread{1.25} % eineinhalb-facher Zeilenabstand

%Kopfzeile
\usepackage{fancyhdr}
\setlength{\headheight}{15pt} %http://nw360.blogspot.de/2006/11/latex-headheight-is-too-small.html



%hypothesen
% http://tex.stackexchange.com/questions/100880/hypotheses-and-subhypotheses-with-ntheorem
\usepackage[hyperref]{ntheorem}
%\makeatletter
  \theoremstyle{break}
  \newtheorem{mainhyp}{Fragestellung}
  \providecommand*{\mainhypautorefname}{Fragestellung}


  \newtheorem{hyp}{Hypothese}
  \providecommand*{\hypautorefname}{Hypothese} %sagt hyperref was es anzeigen soll
  % use \autoref indead of thref (cause you don't want to use different refs)
  % http://tex.stackexchange.com/questions/24568/autoref-names-messed-up-when-defining-certain-ntheorem-theoremstyles/24923#24923
  
%\makeatother

%acronym usage
\usepackage[
  printonlyused,
  withpage 
]{acronym}

\addtokomafont{disposition}{\rmfamily} %überschriften times



\usepackage{lipsum} % um dummy text hinzuklatschen 



\usepackage{afterpage}
% (16:51:58) |Zz|: then try this \usepackage{afterpage}
% (16:52:15) |Zz|: and then put \afterpage{\clearpage} just after that "big figure"
% (16:52:27) |Zz|: it should appear in the next available page, instead at the end of the chapter
% (16:52:34) |Zz|: and that will not disrupt the following figures
% (16:52:41) |Zz|: repeat that trick for each "big figure"

% für unter /labels
\usepackage[list=true,listformat=simple]{subcaption}